Social networks connect people in ways not even imagined when the Internet had just begun to take over the world's population, let alone in the years before. With a single tap of a touchscreen, we are immersed in a world of endless information disseminated by an ever-growing number of blogs, websites, media platforms and social networks. In this sea of information, it is becoming increasingly difficult to find a clear cove.

Before the Internet took hold around the world, the main sources of information were the mainstream media such as the press, TV channels and radio stations, all kinds of banners and leaflets passed from hand to hand, and finally - word of mouth. This type of interaction between the source of information and the target of information (people) is often direct, but can sometimes be through intermediaries. We can imagine a scenario where a leaflet distributor asks other people to help him share the leaflets - not many people will contribute because they may be too busy or simply not interested. Many people may have already seen the leaflet and simply ignore the distributor. Some would certainly appreciate the cause the fliers are promoting and agree to pass it along 

This thesis seeks answers to what happens when someone contributes to share the flyer (tweet) on a Twitter network composed of Croatian users. It collects the available data on Twitter, prepares it for data analysis and aims to quantify the way information is spread on the network in order to identify, among others, the main media and leaflet distributors. The network is analysed by applying different visualisation techniques and data structures to gain insights about the users in this network, the relationships between them, the information they share and the relationships between the information shared.