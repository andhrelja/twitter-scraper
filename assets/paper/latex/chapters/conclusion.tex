Information spread on Twitter is a broad topic that can be analysed in different ways. This thesis describes the methods used to create a \gls{data-platform} that continuously collects Twitter data and creates a dataset that can be used to gain various insights. The methods discussed include the system's development life cycle and the specific software process model that was used to plan and develop a robust \gls{data-pipeline}. \nameref{sec:methods-data-ingestion} and \nameref{sec:methods-data-transformation} methods are then discussed, followed by \nameref{sec:methods-data-analysis} methods and an overview of CI/CD.

Finally, the results provide relevant insights from the dataset created by defining, categorizing, and quantifying the available responses on Twitter from the perspective of time, users, and content, as well as from the perspective of relationships between users and content. They support related work by recognizing the most common category of reaction sharing on Twitter - Retweets. \nameref{subsubsec:methods-data-analysis:graph-analysis} shows how popular content (\texttt{Vatreni}) is shared compared content that is not so popular (\texttt{crypto}). The User sharing popular content is very similar to a mainstream medium - they have a wide reach (number of reverse Incoming links) and are well connected with their audience (clustering coefficient). In contrast, the Users who share not-so-popular content have a short reach and are not well connected with their audience. They represent the leaflet distributors, and their leaflets are not interesting for the people walking next to them. The network that these Users represent resembles a network that connects only the distributors, with very few consumers.

Retweets are useful for answering the question of how information is spread in a network, but they don't answer the question of why it is spread. \nameref{subsubsec:methods-data-analysis:graph-analysis} attempts to provide an answer by using a correlation matrix to identify features attributing to information popularity, but the given answer is short-sighted. The network analysed is a temporal network with one month's worth of data, so the insights gained can be biased by some trends that are unique to the month analysed. Once the analysis is performed on a larger dataset, biases and patterns can be identified and interpreted as an overall picture of information spread.

This thesis serves as a starting point for further research that utilizes the collected data. Existing analysis can be extended or improved in numerous ways, by tracking reaction and content changes over time, or by creating a graph structure in which users represent nodes and a \textit{following} relationship represents a link between users. In such a network, each user shares their own sets of content, which may or may not be similar to the contents of the users around them. By examining the content and comparing it to the content of their \textit{followers} (or \textit{followees}), information paths can be constructed to provide a better view of information spread.
