CI/CD is a software development practice that involves a continuous cycle of building, testing, and deploying software applications. The acronym CI/CD stands for Continuous Integration/Continuous Delivery (or Deployment).

\textbf{CI} refers to continuous integration, which is an automation process for developers. Successful CI means changes to software is regularly built, tested, and merged to a shared \gls{code-repository}. \textbf{CD} is the practice of automatically building, testing, and deploying code changes to a production environment. This allows for a quick and confident delivery of new features and updates to the target system. CI/CD mitigates risks of errors and delays that often occur when deploying code manually.

This \gls{data-platform}'s CI/CD practice uses a \href{https://www.gitkraken.com/learn/git/git-flow}{\textbf{Git Flow}}\footnote{https://www.gitkraken.com/learn/git/git-flow} branching strategy on \gls{github}. It comprises two long-lived branches: \textit{develop} and \texttt{main} (denoted as \textit{production}), short-lived branches like \texttt{feature} and \texttt{hotfix}, driven by \gls{github-actions}. Additionally, a scheduling trigger is set up using \gls{github-actions} to run the \gls{data-pipeline}. This strategy is not being validated by an automated process, it is a verbal convention recommended for a supported, easily managed \gls{code-repository}.

Enhancements and bug fixes are developed inside \texttt{feature} and \texttt{hotfix} branches that developers create prior to starting development. After \textbf{development} and \textbf{\gls{unit-testing}} is completed, a \textbf{pull-request}\footnote{A mechanism for a developer to notify team members that they have completed a development effort \cite{atlassian2022pullrequest}} from the short-lived branch to \textit{develop} is opened on \gls{github}, requesting to merge the short-lived branch to the long-lived branch. Once the \textbf{pull-request} is \textbf{merged}\footnote{"Pull Request Close" action in a \gls{github} repository is a \gls{github} Action event trigger}, an automated process versions the software and tags the code repository using \textbf{\gls{semantic-release}} mechanisms. At the time of writing this thesis, an automated testing process does not exist. Changes are now deployed to \textit{production} by merging \textit{develop} to \texttt{main}.
