Data transformation is the process of converting data from one format or structure into another format or structure. This is often done to make the data more useful or easier to work with for specific purposes, such as analysis or machine learning. Data transformation can involve a wide range of techniques, such as cleaning and preprocessing, normalization, aggregation, and feature extraction. The specific steps involved in a data transformation process will depend on the specific data and the desired end result.

The data transformation process applied to the collected data is designed to be re-runnable in a way that does not affect the stored raw data. This process usually consumes a large amount of time, so it only runs once. The transformed data is then reused throughout the \ref{sec:methods-data-analysis} process. To support the given data requirements, the transformation process filters all collected Users to Croatian users only (\nameref{methods:sdlc-inputs-locations}) and applies other filters (\texttt{statuses\_count > 10}, \texttt{followers\_count > 10}, \texttt{friends\_count > 10} and similar), to ensure the collected Users represent a legitimate sample.

The resulting transformed data is used to create \glspl{data-view} to be used by the \Glspl{data-analyst}.

\begin{code}
\captionof{listing}{Transform User}
\label{subsubsec:sdlc-analysis-data-transformations:transform-user}
\begin{minted}[frame=single,
               framesep=3mm,
               tabsize=2]{js}
{
    "name": "HNS",
    "screen_name": "HNS_CFF",
    "location": "Croatia",
    "is_croatian": true,
    "description": "Službeni Twitter profil Hrvatskog"
                   "nogometnog saveza Croatian Football"
                   "Federation official Twitter feed."
                   "#HNS #Obitelj #Family",
    "followers_count": 253625,
    "friends_count": 117,
    "statuses_count": 19209,
    "created_at": "2022-08-30T05:56:36+00:00"
}
\end{minted}
\end{code}

The User object contains a small number of transformations compared to the Tweet object. The following attributes and transformations are applied to the User object:

\begin{itemize}
    \item \texttt{location}: \underline{extract} city name text value if it is included in the original value, otherwise set to \texttt{Hrvatska}
    \item \texttt{is\_croatian}: \underline{create} boolean value to indicate whether or not the user is from Croatia 
    \item \texttt{description}: User provided profile description
    \item \texttt{followers\_count}: number of users following this User
    \item \texttt{friends\_count}: number of users this User follows
    \item \texttt{statuses\_count}: total number of published Tweets since \texttt{created\_at}
    \item \texttt{created\_at}: \underline{evaluate} string to date-time object; represents profile creation date and time
\end{itemize}

\clearpage
\begin{code}
\captionof{listing}{Transform Tweet}
\label{subsubsec:sdlc-analysis-data-transformations:transform-tweet}
\begin{minted}[frame=single,
               framesep=3mm,
               tabsize=2]{js}
{
    "created_at": "2022-11-22 10:00:18+00:00",
    "created_at_year": 2022,
    "created_at_month": 11,
    "created_at_week": 47,
    "created_at_day": 22,
    "full_text": "VATRENI IS LISTED"
                 "\n\nThis is truly a historic moment"
                 "because #VATRENI is so much more "
                 "than just a token. Become a part"
                 "of the greatest fan story and enjoy"
                 "all kinds of benefits.\n\n#VATRENI"
                 "token is now live at @gate_io \n\n"
                 "Get it here: "
                 "https://t.co/sN8mWtUac6"
                 "https://t.co/XfqNfwUeCS",
    "hashtags": ["VATRENI", "VATRENI"],
    "user_mentions": ["gate_io"],
    "is_retweet": true,
    "retweet_count": 15,
    "retweet_created_at": "2022-11-22 10:29:47+00:00",
    "retweet_from_tweet_id": 1594994170858463232,
    "retweet_from_user_name": "vatreni_token",
    "retweet_timedelta_sec": 960,
    "is_reply": false,
    "reply_to_tweet_id": null,
    "reply_to_user_name": null,
    "is_quote": false,
    "favorite_count": 41,
    "possibly_sensitive": false,
    "lang": "en",
    "transform_date": "2022-11-22"
}
\end{minted}
\end{code}

The Tweet object contains a large number of transformations. The following attributes and transformations are applied to the Tweet object:

\begin{itemize}
    \item \texttt{created\_at\_year}: \underline{extract} year number from \texttt{created\_at}
    \item \texttt{created\_at\_month}: \underline{extract} month number from \texttt{created\_at}
    \item \texttt{created\_at\_week}: \underline{extract} week number from \texttt{created\_at}
    \item \texttt{created\_at\_day}: \underline{extract} day number from \texttt{created\_at}
    \item \texttt{hashtags}: \underline{extract} hashtag text from \texttt{entities.hashtags}
    \item \texttt{user\_mentions}: \underline{extract} mentioned user's screen\_name from \\ 
    \texttt{entities.user\_mentions}
    \item \texttt{is\_retweet}: \underline{create} boolean value based on existence of \\
    \texttt{retweeted\_status}
    \item \texttt{retweet\_count}: \underline{create} numeric value based on all other Tweet objects where their \texttt{retweeted\_status.id} equals this Tweet object's \texttt{id}
    \item \texttt{retweet\_created\_at}: \underline{extract} date-time object from \\
    \texttt{retweeted\_status.created\_at}
    \item \texttt{retweet\_from\_tweet\_id}: \underline{extract} numeric Tweet identifier from \\
    \texttt{retweeted\_status.id}
    \item \texttt{retweet\_from\_user\_name}: \underline{extract} text User identifier from \\
    \texttt{retweeted\_status.user.user\_id}
    \item \texttt{retweet\_timedelta\_sec}: \underline{create} \href{https://pandas.pydata.org/docs/reference/api/pandas.Timedelta.html}timedelta\footnote{https://pandas.pydata.org/docs/reference/api/pandas.Timedelta.html} value based on the difference between \texttt{retweet\_created\_at} and \texttt{created\_at}
    \item \texttt{is\_reply}: \underline{create} boolean value based on existence of \\
    \texttt{in\_reply\_to\_status\_id}
    \item \texttt{reply\_to\_tweet\_id}: \underline{rename} \texttt{in\_reply\_to\_status\_id}
    \item \texttt{reply\_to\_user\_name}: \underline{rename} \texttt{in\_reply\_to\_screen\_name}
    \item \texttt{is\_quote}: \underline{rename} \texttt{is\_quote\_status}
    \item \texttt{lang}: \underline{apply} a language detection function using \href{https://github.com/saffsd/langid.py}{langid}\footnote{https://github.com/saffsd/langid.py} if the original value was undefined
    \item \texttt{transform\_date}: \underline{create} text value based on the transformation date
\end{itemize}
